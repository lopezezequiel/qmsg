%% Generated by Sphinx.
\def\sphinxdocclass{report}
\documentclass[letterpaper,10pt,spanish]{sphinxmanual}
\ifdefined\pdfpxdimen
   \let\sphinxpxdimen\pdfpxdimen\else\newdimen\sphinxpxdimen
\fi \sphinxpxdimen=.75bp\relax

\usepackage[utf8]{inputenc}
\ifdefined\DeclareUnicodeCharacter
 \ifdefined\DeclareUnicodeCharacterAsOptional\else
  \DeclareUnicodeCharacter{00A0}{\nobreakspace}
\fi\fi
\usepackage{cmap}
\usepackage[T1]{fontenc}
\usepackage{amsmath,amssymb,amstext}
\usepackage{babel}
\usepackage{times}
\usepackage[Sonny]{fncychap}
\usepackage[dontkeepoldnames]{sphinx}

\usepackage{geometry}

% Include hyperref last.
\usepackage{hyperref}
% Fix anchor placement for figures with captions.
\usepackage{hypcap}% it must be loaded after hyperref.
% Set up styles of URL: it should be placed after hyperref.
\urlstyle{same}
\addto\captionsspanish{\renewcommand{\contentsname}{Contents:}}

\addto\captionsspanish{\renewcommand{\figurename}{Figura}}
\addto\captionsspanish{\renewcommand{\tablename}{Tabla}}
\addto\captionsspanish{\renewcommand{\literalblockname}{Lista}}

\addto\extrasspanish{\def\pageautorefname{página}}

\setcounter{tocdepth}{1}



\title{qmsg Documentation}
\date{22 de junio de 2017}
\release{1.0}
\author{los illuminatis}
\newcommand{\sphinxlogo}{\vbox{}}
\renewcommand{\releasename}{Versión}
\makeindex

\begin{document}
\ifnum\catcode`\"=\active\shorthandoff{"}\fi
\maketitle
\sphinxtableofcontents
\phantomsection\label{\detokenize{index::doc}}



\chapter{Tests}
\label{\detokenize{modules/tests:tests}}\label{\detokenize{modules/tests:documentacion-de-qmsg}}\label{\detokenize{modules/tests::doc}}\index{MessageTestCase (clase en apirestv1.tests)}

\begin{fulllineitems}
\phantomsection\label{\detokenize{modules/tests:apirestv1.tests.MessageTestCase}}\pysigline{\sphinxstrong{class }\sphinxcode{apirestv1.tests.}\sphinxbfcode{MessageTestCase}}~\index{test\_MITM() (método de apirestv1.tests.MessageTestCase)}

\begin{fulllineitems}
\phantomsection\label{\detokenize{modules/tests:apirestv1.tests.MessageTestCase.test_MITM}}\pysiglinewithargsret{\sphinxbfcode{test\_MITM}}{}{}
Prueba que no se pueda setear el id de un mensaje a traves de
la api rest

\end{fulllineitems}

\index{test\_bad\_request() (método de apirestv1.tests.MessageTestCase)}

\begin{fulllineitems}
\phantomsection\label{\detokenize{modules/tests:apirestv1.tests.MessageTestCase.test_bad_request}}\pysiglinewithargsret{\sphinxbfcode{test\_bad\_request}}{}{}
Prueba que devuelva un http 400 si se envían datos inválidos

\end{fulllineitems}

\index{test\_collisions() (método de apirestv1.tests.MessageTestCase)}

\begin{fulllineitems}
\phantomsection\label{\detokenize{modules/tests:apirestv1.tests.MessageTestCase.test_collisions}}\pysiglinewithargsret{\sphinxbfcode{test\_collisions}}{}{}
Valida que no haya colisiones entre ids autogenerados de
mensajes que se crean en un corto periodo de tiempo

\end{fulllineitems}

\index{test\_id() (método de apirestv1.tests.MessageTestCase)}

\begin{fulllineitems}
\phantomsection\label{\detokenize{modules/tests:apirestv1.tests.MessageTestCase.test_id}}\pysiglinewithargsret{\sphinxbfcode{test\_id}}{}{}
Valida que se genere un id válido cuando se crea un mensaje.
formato válido: 8 caracteres hexadecimales.
regexp: \textasciicircum{}{[}0-9a-f{]}\{8\}\$

\end{fulllineitems}

\index{test\_message\_params() (método de apirestv1.tests.MessageTestCase)}

\begin{fulllineitems}
\phantomsection\label{\detokenize{modules/tests:apirestv1.tests.MessageTestCase.test_message_params}}\pysiglinewithargsret{\sphinxbfcode{test\_message\_params}}{}{}
Prueba aceptación de parámetros correctos

\end{fulllineitems}

\index{test\_rest() (método de apirestv1.tests.MessageTestCase)}

\begin{fulllineitems}
\phantomsection\label{\detokenize{modules/tests:apirestv1.tests.MessageTestCase.test_rest}}\pysiglinewithargsret{\sphinxbfcode{test\_rest}}{}{}
Prueba la api rest para Message

\end{fulllineitems}

\index{test\_unique\_constraint() (método de apirestv1.tests.MessageTestCase)}

\begin{fulllineitems}
\phantomsection\label{\detokenize{modules/tests:apirestv1.tests.MessageTestCase.test_unique_constraint}}\pysiglinewithargsret{\sphinxbfcode{test\_unique\_constraint}}{}{}
Prueba que no permita crear mensajes con ids duplicados

\end{fulllineitems}


\end{fulllineitems}



\chapter{Vistas}
\label{\detokenize{modules/views:vistas}}\label{\detokenize{modules/views::doc}}\index{index() (en el módulo apirestv1.views)}

\begin{fulllineitems}
\phantomsection\label{\detokenize{modules/views:apirestv1.views.index}}\pysiglinewithargsret{\sphinxcode{apirestv1.views.}\sphinxbfcode{index}}{\emph{request}}{}
Renderiza la página princial

\end{fulllineitems}

\index{post\_message() (en el módulo apirestv1.views)}

\begin{fulllineitems}
\phantomsection\label{\detokenize{modules/views:apirestv1.views.post_message}}\pysiglinewithargsret{\sphinxcode{apirestv1.views.}\sphinxbfcode{post\_message}}{\emph{request}}{}
Recibe un mensaje codificado en json, lo guarda en base de datos
y lo retorna con el id seteado

\end{fulllineitems}

\index{get\_message() (en el módulo apirestv1.views)}

\begin{fulllineitems}
\phantomsection\label{\detokenize{modules/views:apirestv1.views.get_message}}\pysiglinewithargsret{\sphinxcode{apirestv1.views.}\sphinxbfcode{get\_message}}{\emph{request}, \emph{id}}{}
Devuelve un mensaje y lo elimina de base de datos

\end{fulllineitems}



\chapter{Modelo}
\label{\detokenize{modules/models:module-apirestv1.models}}\label{\detokenize{modules/models::doc}}\label{\detokenize{modules/models:modelo}}\index{apirestv1.models (módulo)}\index{Message (clase en apirestv1.models)}

\begin{fulllineitems}
\phantomsection\label{\detokenize{modules/models:apirestv1.models.Message}}\pysiglinewithargsret{\sphinxstrong{class }\sphinxcode{apirestv1.models.}\sphinxbfcode{Message}}{\emph{*args}, \emph{**kwargs}}{}
Representa un mensaje que tiene un id de 8 digitos hexadecimales
y un texto de hasta 2000 caracteres

\end{fulllineitems}



\chapter{Serializadores}
\label{\detokenize{modules/serializers:module-apirestv1.serializers}}\label{\detokenize{modules/serializers::doc}}\label{\detokenize{modules/serializers:serializadores}}\index{apirestv1.serializers (módulo)}\index{MessageSerializer (clase en apirestv1.serializers)}

\begin{fulllineitems}
\phantomsection\label{\detokenize{modules/serializers:apirestv1.serializers.MessageSerializer}}\pysiglinewithargsret{\sphinxstrong{class }\sphinxcode{apirestv1.serializers.}\sphinxbfcode{MessageSerializer}}{\emph{instance=None}, \emph{data=\textless{}class rest\_framework.fields.empty\textgreater{}}, \emph{**kwargs}}{}
Serializador/Deserialiador de mensajes
El campo id es de solo lectura

\end{fulllineitems}



\chapter{Indices and tables}
\label{\detokenize{index:indices-and-tables}}\begin{itemize}
\item {} 
\DUrole{xref,std,std-ref}{genindex}

\item {} 
\DUrole{xref,std,std-ref}{modindex}

\item {} 
\DUrole{xref,std,std-ref}{search}

\end{itemize}


\renewcommand{\indexname}{Índice de Módulos Python}
\begin{sphinxtheindex}
\def\bigletter#1{{\Large\sffamily#1}\nopagebreak\vspace{1mm}}
\bigletter{a}
\item {\sphinxstyleindexentry{apirestv1.models}}\sphinxstyleindexpageref{modules/models:\detokenize{module-apirestv1.models}}
\item {\sphinxstyleindexentry{apirestv1.serializers}}\sphinxstyleindexpageref{modules/serializers:\detokenize{module-apirestv1.serializers}}
\end{sphinxtheindex}

\renewcommand{\indexname}{Índice}
\printindex
\end{document}